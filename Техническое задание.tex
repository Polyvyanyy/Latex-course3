\documentclass[russian,utf8,pointsection,simple,14pt]{eskdtext}
\usepackage[T2A]{fontenc}
\usepackage[utf8]{inputenc}
\usepackage[russian]{babel}
\DeclareTextSymbol{\No}{T2A}{"9D}
\usepackage{eskdchngsheet}
\usepackage[table,xcdraw]{xcolor}
\usepackage{amstext}
\usepackage{amsmath}
\usepackage{listings}
%\usepackage[unicode]{hyperref}
\usepackage{ragged2e}
\usepackage{multirow}
\usepackage{hhline}
\ESKDsectStyle{section}{\rmfamily}

\newcommand{\specialcell}[2][c]{%
	\begin{tabular}[#1]{@{}c@{}}#2\end{tabular}}
\begin{document}
\maketitle
\tableofcontents
\newpage
\section{Анализ технического задания}
Заданием на курсовую работу является разработка ЛВС\footnote{Локальная вычислительная сеть.} сети организации.
\medskip
\point Исходные данные для проектирования:
\begin{itemize}
\item Минимальное количество оборудования --- 20;
\item Использование сервера баз данных;
\item Возможность взаимодействия с интернетом;
\item Количество зданий --- 2;
\item Тип базы данных - реляционная\footnote{База данных, основанная на реляционной модели данных.};
\item Количество таблиц\footnote{Количество таблиц в базе данных.} --- не меньше 5;
\end{itemize}

Обьектом для реализации сети будем считать Театр. Данная организация занимает два одноэтажных здания, расположенных в одном городе. Она занимает 7 комнат, которые предназначены для полного функционирования организации.
\medskip 
\point Описание работы каждого отдела:
\begin{enumerate}
	\item Организация мероприятий:
	
		Работники данного отдела занимаются организацией мероприятий и, соответственно, формированием таблицы мероприятий в базе данных для регистрации посетителей на них. 

	\item Регистрация посетителей на мероприятие:

Работники данного отдела занимаются регистрацией посетителей на конкретные мероприятия и, соответственно, формированием таблицы для базы данных посетителей для каждого мероприятия.

    \item Отдел обслуживания информационной системы:
    
Работники данного отдела занимаются обслуживанием информационной системы организации и, соответственно, обслуживание сервера базы данных.

\end{enumerate}

\point Определения количества компьютеров в каждом отделе в таблице \ref{t:1}
\begin{table}[h]
    \caption{Распределение компьютеров по отделам.}
	\label{t:1}
	\begin{tabular}{|c|c|c|c|}
		\hline

		\multicolumn{1}{|p{5cm}|}{№ компьютера}&
		\multicolumn{1}{p{4cm}|}{Название отдела}&
		\multicolumn{1}{p{4cm}|}{Задача}\\\hhline{|===|}
		Театр №1 - №9 & \specialcell{Организация \\ мероприятий} & Рабочая станция\\\hline
		Театр №10 - №27 & \specialcell{Регистрация посетителей \\ на мероприятие} & Рабочая станция\\\hline
		Театр №28 - №29 & \specialcell{Отдел обслуживания \\ информационной системы }& Рабочая станция и сервер\\\hline
	\end{tabular}
\end{table}
\point Описание задач выполняемых организацией  с помощью ЛВС
\medskip

Исходя из задания на курсовую работу определим основные задачи для предприятия решаемые непосредственным использованием локальной вычислительной сети (ЛВС):
\begin{enumerate}
	\item Распределенная база данных (содержит таблицы посетителей и мероприятий для организации работы театра);
	\item Регистрация и организация, то есть формирование таблиц базы данных для обслуживания посетителей;
	\item Выход в интернет через провайдера интернет услуг;
	
\end{enumerate}

\point Безопасность ЛВС
\medskip

Так как наша созданная ЛВС будет взаимодействовать с глобальной ИВС, поэтому для обеспечения информационной защиты данных будут установлены два межсетевых экрана. А также непосредственно ограничение доступа к ИВС в некоторых отделах предприятия.

\end{document}