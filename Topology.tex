% !TeX spellcheck = en_US
\documentclass[russian,utf8,pointsection,simple,14pt]{eskdtext}
\usepackage[T2A]{fontenc}
\usepackage[utf8]{inputenc}
\usepackage[russian]{babel}
\DeclareTextSymbol{\No}{T2A}{"9D}
\usepackage{eskdchngsheet}
\usepackage[table,xcdraw]{xcolor}
\usepackage{amstext}
\usepackage{amsmath}
\usepackage{listings}
\usepackage[unicode]{hyperref}
\usepackage{ragged2e}
\usepackage{multirow}
\ESKDdocName{текст}
\ESKDsectStyle{section}{\rmfamily}
\newcommand{\specialcell}[2][c]{%
	\begin{tabular}[#1]{@{}c@{}}#2\end{tabular}}
\begin{document}
	\maketitle
	\newpage
	\tableofcontents
	\newpage
	\section{Выбор и обоснование топологии сети}
	Сетевая топология — это конфигурация графа, вершинам которого соответствуют конечные узлы сети (компьютеры) и коммуникационное оборудование (маршрутизаторы), а рёбрам — физические или информационные связи между вершинами. 
	
	\point Топологии делятся на 2 типа:
	\begin{itemize}
		\item  Полносвязная
		\item Неполносвязная
		\item Смешанная топология
		
	\end{itemize}

Полносвязная топология - это сеть, в которой каждый компьютер связан со другими компьютерами сети напрямую. Такая сеть неэффективна, потому как каждый компьютер должен иметь достаточно большое количество портов.

Неполносвязная топология в отличие от полносвязных, может передавать данные через дополнительные узлы.
\point Неполносвязные топологии и их общие представления:
\begin{enumerate}
\item Шина --- представляет собой общий кабель, к которому подсоединены все рабочие станции, преимуществом являются: меньший расход кабеля, отказ одно узла не влияет на работу другого, легко настраиваемая сеть. Недостатком являются: разрыв одного кабеля влияет на работу всей сети, ограничение по длине кабеля и кол-ву рабочих станций, низкая производительность из-за разделения канала кабеля на несколько рабочих станций.
\item Звезда --- каждая рабочай станция подсоединяется витой парой к концентратору, который обеспечивает параллельное соединение станций, преимуществом являются: легкость в подключении новой станции, возможность централизованного управления, устойчивость к неисправностям отдельных станций. Недостатком являются: отказ концентратора влияет на работу всей сети, достаточно большой расход кабеля.
\item Кольцо --- все узлы соединены каналами связи в неразрывное кольцо. Начав движение из одной точки, данные попадают на его начало. Сеть очень легко создавать и настраивать, но повреждение линии связи приводит к неработоспособности всей сети, в чистом виде данная топология не применяется из-за своей ненадеждности.
\end{enumerate}

Смешанная топология --- это топология, использующаяся в крупных сетях с разными связями между рабочими станциями. В таких сетях можно выделить подсети, которые будут иметь типовую топологию.

Для решения требований технического задания было решено выбрать смешанную топологию, которая будет состоять из топологии "Звезда" и топологии "Шина". Сеть с данной топологией будет централизованной, то есть будет иметь центральный узел, через который проходит весь трафик сети --- концентратор.

Для обеспечения информационной безопасности рабочие станции разделены на подсети с помощью маски подсети.


\end{document}