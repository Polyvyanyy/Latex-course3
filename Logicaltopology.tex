% !TeX spellcheck = en_US
\documentclass[russian,utf8,pointsection,simple,14pt]{eskdtext}
\usepackage[T2A]{fontenc}
\usepackage[utf8]{inputenc}
\usepackage[russian]{babel}
\DeclareTextSymbol{\No}{T2A}{"9D}
\usepackage{eskdchngsheet}
\usepackage[table,xcdraw]{xcolor}
\usepackage{amstext}
\usepackage{amsmath}
\usepackage{listings}
\usepackage[unicode]{hyperref}
\usepackage{ragged2e}
\usepackage{multirow}
\ESKDdocName{текст}
\ESKDsectStyle{section}{\rmfamily}
\newcommand{\specialcell}[2][c]{%
	\begin{tabular}[#1]{@{}c@{}}#2\end{tabular}}
\begin{document}
	\maketitle
	\newpage
	\tableofcontents
	\newpage
	\section{Логическая топология ЛВС}
	
	Логическая топология определяет реальные пути движения данных по используемой физической топологией. Она описывает пути передачи потоков данных между сетевыми устройствами, определяет правила передачи данных в существующей среде передачи.
	
	Для решения требований технического задания было решено выбрать логическую смешанную топологию.
	
	Логическая топология приведена в приложении Б.
\end{document}