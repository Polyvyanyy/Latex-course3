\documentclass[russian,utf8,pointsection,simple,14pt]{eskdtext}
\usepackage[T2A]{fontenc}
\usepackage[utf8]{inputenc}
\usepackage[russian]{babel}
\DeclareTextSymbol{\No}{T2A}{"9D}
\usepackage{eskdchngsheet}
\usepackage[table,xcdraw]{xcolor}
\usepackage{amstext}
\usepackage{amsmath}
\usepackage{listings}
\usepackage[unicode]{hyperref}
\usepackage{ragged2e}
\usepackage{multirow}
\usepackage{hhline}
\ESKDdocName{Разработка ЛВС и базы данных для предприятия Театр}
\ESKDsectStyle{section}{\rmfamily}
\newcommand{\specialcell}[2][c]{%
	\begin{tabular}[#1]{@{}c@{}}#2\end{tabular}}
\begin{document}
	
	\maketitle
	\maketitle
	\maketitle
	\newpage
	\tableofcontents
	
	\newpage
	\section{Введение}
	\newpage
	
	\section{Анализ технического задания}
	Заданием на курсовую работу является разработка ЛВС\footnote{Локальная вычислительная сеть.} сети организации.
	\medskip
	\point Исходные данные для проектирования:
	\begin{itemize}
		\item Минимальное количество оборудования --- 20;
		\item Использование сервера баз данных;
		\item Возможность взаимодействия с интернетом;
		\item Количество зданий --- 2;
		\item Тип базы данных - реляционная\footnote{База данных, основанная на реляционной модели данных.};
		\item Количество таблиц\footnote{Количество таблиц в базе данных.} --- не меньше 5;
	\end{itemize}
	
	Обьектом для реализации сети будем считать Театр. Данная организация занимает два одноэтажных здания, расположенных в одном городе. Она занимает 7 комнат, которые предназначены для полного функционирования организации.
	\medskip 
	\point Описание работы каждого отдела:
	\begin{enumerate}
		\item Организация мероприятий:
		
		Работники данного отдела занимаются организацией мероприятий и, соответственно, формированием таблицы мероприятий в базе данных для регистрации посетителей на них. 
		
		\item Регистрация посетителей на мероприятие:
		
		Работники данного отдела занимаются регистрацией посетителей на конкретные мероприятия и, соответственно, формированием таблицы для базы данных посетителей для каждого мероприятия.
		
		\item Отдел обслуживания информационной системы:
		
		Работники данного отдела занимаются обслуживанием информационной системы организации и, соответственно, обслуживание сервера базы данных.
		
	\end{enumerate}
	
	\point Определения количества компьютеров в каждом отделе в таблице \ref{t:1}
	\begin{table}[h]
		\caption{Распределение компьютеров по отделам.}
		\label{t:1}
		\begin{tabular}{|c|c|c|c|}
			\hline
			
			\multicolumn{1}{|p{5cm}|}{№ компьютера}&
			\multicolumn{1}{p{4cm}|}{Название отдела}&
			\multicolumn{1}{p{4cm}|}{Задача}\\\hhline{|===|}
			Театр №1 - №9 & \specialcell{Организация \\ мероприятий} & Рабочая станция\\\hline
			Театр №10 - №27 & \specialcell{Регистрация посетителей \\ на мероприятие} & Рабочая станция\\\hline
			Театр №28 - №29 & \specialcell{Отдел обслуживания \\ информационной системы }& Рабочая станция и сервер\\\hline
		\end{tabular}
	\end{table}
	\point Описание задач выполняемых организацией  с помощью ЛВС
	\medskip
	
	Исходя из задания на курсовую работу определим основные задачи для предприятия решаемые непосредственным использованием локальной вычислительной сети (ЛВС):
	\begin{enumerate}
		\item Распределенная база данных (содержит таблицы посетителей и мероприятий для организации работы театра);
		\item Регистрация и организация, то есть формирование таблиц базы данных для обслуживания посетителей;
		\item Выход в интернет через провайдера интернет услуг;
		
	\end{enumerate}
	
	\point Безопасность ЛВС
	\medskip
	
	Так как наша созданная ЛВС будет взаимодействовать с глобальной ИВС, поэтому для обеспечения информационной защиты данных будут установлены два межсетевых экрана. А также непосредственно ограничение доступа к ИВС в некоторых отделах предприятия.
	\newpage
	
	\section{Выбор способов передачи данных}
	\point При выборе физической среды передачи данных учитываются следующие показатели:
	\begin{itemize}
		\item Стоимость оборудования, монтажа и обслуживания;
		\item Максимальная скорость передачи информации;
		\item Ограничения на максимальную длину кабеля;
		\item Безопасность и надежность функционирования сети.
	\end{itemize}
	
	\point Для подключения компьютеров между собой будет использоваться стандарт беспроводной локальной связи Wi-Fi 4\footnote{ IEEE 802.11n — версия стандарта 802.11 для сетей Wi-Fi, появившаяся в 2009 году. } на частоте 2.4 ГГц, его характеристики:
	\begin{itemize}
		\item Максимальная скорость передачи данных 300 Мбит/с;
		\item Максимальное расстояние от передатчика 150 метров на открытой местности;
		\item Легкость монтажа и наращивания ЛВС;
		\item Низкая стоимость. 
	\end{itemize}
	
	\point Способы передачи данных
	
	\medskip
	Непосредственно главным моментом в проектировании сети является выбор способов передачи данных. Способ передачи определяется сетевой технологией, на основе которой построена ЛВС. Оптимальным решением для данной сети являются технологии Fast Ethernet\footnote{Fast Ethernet (FE) — общее название для набора стандартов передачи данных в компьютерных сетях по технологии Ethernet со скоростью до 100 Мбит/с.} и Wi-Fi. Они полностью подходят для данной ЛВС по скорости передачи данных, безопасности передачи информации, обратной совместимости с предшествующим им технологиям. Для построения сети достаточно иметь по одному сетевому адаптеру поддерживающие стандарты Wi-Fi для каждого компьютера и один маршрутизатор. Данные технологии позволяют иметь ЛВС хорошую расширяемость, низкую стоимость, простота настройки и эксплуатации. Работа стандартов Wi-Fi основана на передаче идентификатора сети SSID с помощью пакетов на скорости 0,1 Мбит/с каждые 100 мс. Потому 0,1 Мбит/с --- наименьшая скорость передачи данных Wi-Fi. С помощью SSID клиент может выяснить о возможности подключения к точке доступа.
	Защита канала передачи данных с помощью Wi-Fi основана на методах шифрования:
	\begin{itemize}
		\item WEP\footnote{Wired Equaivalent Privacy} --- это первый стандарт защиты Wi-Fi, на самом деле не дает защиты по сравнению с проводными сетями, так как имеет множество уязвимостей и взламывается множеством разных способов, что из-за расстояния покрываемого передатчиком, делает данные более уязвимыми. Данный протокол обеспечивает защиту канала передачи данных только на короткое время, спустя которое любую передачу данных можно взломать вне зависимости от сложности пароля --- пароли в WEP либо 40 либо 104 бита, что есть крайне короткая комбинация. Все эти недостатки имеются из-за времени создания WEP --- конец 90-х годов и потому IEEE\footnote{Институт инженеров электротехники и электроники — IEEE (англ. Institute of Electrical and Electronics Engineers) (I triple E — «Ай трипл и») — международная некоммерческая ассоциация специалистов в области техники, мировой лидер в области разработки стандартов по радиоэлектронике, электротехнике и аппаратному обеспечению вычислительных систем и сетей.} в 2004 году обьявили WEP устаревшим.
		\item WPA\footnote{Wi-Fi Protected Access}  --- второе поколение защищенных протоколов Wi-Fi. Совершенно иной уровень защиты каналов данных в сравнении с WEP. Длина пароля произвольная, от 8 до 63 байт. Поддерживает различные алгоритмы шифрования данных после авторизации в сети. WPA в главной степени отличается от WEP тем, что шифрует данные каждого клиента по отдельности --- после авторизации генерируется временный ключ PTK который используется для кодирования передачи данных конкретного клиента.
		\item WPS\footnote{Wi-Fi Protected Setup} --- протокол, позволяющий для авторизации вместо пароля использовать кнопку на роутере для подключения. При выпуске этого протокола была фундаментальная уязвимость --- WPS позволяет подключиться к точке доступа по 8-ми символьному коду, состоящему из цифр. Но из-за ошибки нужно подобрать всего лишь 4 из них.
	\end{itemize}
\newpage

\section{Разработка информационной модели ЛВС}
\newpage

	\section{Выбор и обоснование топологии сети}
Сетевая топология — это конфигурация графа, вершинам которого соответствуют конечные узлы сети (компьютеры) и коммуникационное оборудование (маршрутизаторы), а рёбрам — физические или информационные связи между вершинами. 

\point Топологии делятся на 2 типа:
\begin{itemize}
	\item  Полносвязная
	\item Неполносвязная
	\item Смешанная топология
	
\end{itemize}

Полносвязная топология - это сеть, в которой каждый компьютер связан со другими компьютерами сети напрямую. Такая сеть неэффективна, потому как каждый компьютер должен иметь достаточно большое количество портов.

Неполносвязная топология в отличие от полносвязных, может передавать данные через дополнительные узлы.
\point Неполносвязные топологии и их общие представления:
\begin{enumerate}
	\item Шина --- представляет собой общий кабель, к которому подсоединены все рабочие станции, преимуществом являются: меньший расход кабеля, отказ одно узла не влияет на работу другого, легко настраиваемая сеть. Недостатком являются: разрыв одного кабеля влияет на работу всей сети, ограничение по длине кабеля и кол-ву рабочих станций, низкая производительность из-за разделения канала кабеля на несколько рабочих станций.
	\item Звезда --- каждая рабочай станция подсоединяется витой парой к концентратору, который обеспечивает параллельное соединение станций, преимуществом являются: легкость в подключении новой станции, возможность централизованного управления, устойчивость к неисправностям отдельных станций. Недостатком являются: отказ концентратора влияет на работу всей сети, достаточно большой расход кабеля.
	\item Кольцо --- все узлы соединены каналами связи в неразрывное кольцо. Начав движение из одной точки, данные попадают на его начало. Сеть очень легко создавать и настраивать, но повреждение линии связи приводит к неработоспособности всей сети, в чистом виде данная топология не применяется из-за своей ненадеждности.
\end{enumerate}

Смешанная топология --- это топология, использующаяся в крупных сетях с разными связями между рабочими станциями. В таких сетях можно выделить подсети, которые будут иметь типовую топологию.

Для решения требований технического задания было решено выбрать смешанную топологию, которая будет состоять из топологии "Звезда" и топологии "Шина". Сеть с данной топологией будет централизованной, то есть будет иметь центральный узел, через который проходит весь трафик сети --- концентратор.

Для обеспечения информационной безопасности рабочие станции разделены на подсети с помощью маршрутизатора с технологией VLAN.

\newpage
	\section{Физическая топология ЛВС}
\
Физическая топология ЛВС описывает реально использующиеся способы организации физических соединений различного сетевого оборудования(кабели, разъемы, способы подключения сетевого оборудования).

Для решения требований технического задания было решено выбрать физическую смешанную топологию.

Физическая топология приведена в приложении А.
\newpage

	\section{Логическая топология ЛВС}

Логическая топология определяет реальные пути движения данных по используемой физической топологией. Она описывает пути передачи потоков данных между сетевыми устройствами, определяет правила передачи данных в существующей среде передачи.

Для решения требований технического задания было решено выбрать логическую смешанную топологию.

Логическая топология приведена в приложении Б.

\newpage

\section{Расчет и выбор объединения ЛВС}
Выбрав смешанную топологию, встала необходимость выбрать элементы объединения рабочих станций сети, просчитать их расположение, и взаимосвязи между всеми объектами сети.

В данной курсовой работе связь рабочих станций сети будет осуществляться с помощью двух пяти портовых коммутаторов, одного маршрутизатора, четырех Wi-Fi роутеров. Данный выбор обусловлен необходимостью разделения сети на подсети, обеспечения информационной безопасности, облегчения добавления новых рабочих машин посредством технологии Wi-Fi. Стоимость сети складывается из стоимости всего оборудования.

Произведенный расчет представлен в приложении В.




\end{document}