\documentclass[russian,utf8,pointsection,simple,14pt]{eskdtext}
\usepackage[T2A]{fontenc}
\usepackage[utf8]{inputenc}
\usepackage[russian]{babel}
\DeclareTextSymbol{\No}{T2A}{"9D}
\usepackage{eskdchngsheet}
\usepackage[table,xcdraw]{xcolor}
\usepackage{amstext}
\usepackage{amsmath}
\usepackage{listings}
\usepackage[unicode]{hyperref}
\usepackage{ragged2e}
\usepackage{multirow}
\ESKDdocName{Разработка ЛВС и базы данных для предприятия Театр}
\ESKDsectStyle{section}{\rmfamily}
\newcommand{\specialcell}[2][c]{%
	\begin{tabular}[#1]{@{}c@{}}#2\end{tabular}}
\begin{document}
	
	\maketitle
	\maketitle
	\maketitle
	\newpage
	\tableofcontents
	
	\newpage
	
	\section{Расчет и выбор объединения ЛВС}
	Выбрав смешанную топологию, встала необходимость выбрать элементы объединения рабочих станций сети, просчитать их расположение, и взаимосвязи между всеми объектами сети.
	
	В данной курсовой работе связь рабочих станций сети будет осуществляться с помощью двух пяти портовых коммутаторов, одного маршрутизатора, четырех Wi-Fi роутеров. Данный выбор обусловлен необходимостью разделения сети на подсети, обеспечения информационной безопасности, облегчения добавления новых рабочих машин посредством технологии Wi-Fi. Стоимость сети складывается из стоимости всего оборудования.
	
	Произведенный расчет представлен в приложении.
	
\end{document}