\documentclass[russian,utf8,pointsection,simple,14pt]{eskdtext}
\usepackage[T2A]{fontenc}
\usepackage[utf8]{inputenc}
\usepackage[russian]{babel}
\DeclareTextSymbol{\No}{T2A}{"9D}
\usepackage{eskdchngsheet}
\usepackage[table,xcdraw]{xcolor}
\usepackage{amstext}
\usepackage{amsmath}
\usepackage{listings}
\usepackage[unicode]{hyperref}
\usepackage{ragged2e}
\usepackage{multirow}
\ESKDsectStyle{section}{\rmfamily}

\newcommand{\specialcell}[2][c]{%
	\begin{tabular}[#1]{@{}c@{}}#2\end{tabular}}
\begin{document}
\tableofcontents
\newpage
\section{Способы передачи данных}
\point При выборе физической среды передачи данных учитываются следующие показатели:
\begin{itemize}
\item Стоимость оборудования, монтажа и обслуживания;
\item Максимальная скорость передачи информации;
\item Ограничения на максимальную длину кабеля;
\item Безопасность и надежность функционирования сети.
\end{itemize}

\point Для подключения компьютеров между собой будет использоваться стандарт беспроводной локальной связи Wi-Fi 4\footnote{ IEEE 802.11n — версия стандарта 802.11 для сетей Wi-Fi, появившаяся в 2009 году. } на частоте 2.4 ГГц, его характеристики:
\begin{itemize}
\item Максимальная скорость передачи данных 300 Мбит/с;
\item Максимальное расстояние от передатчика 150 метров на открытой местности;
\item Легкость монтажа и наращивания ЛВС;
\item Низкая стоимость. 
\end{itemize}

\point Способы передачи данных

\medskip
Непосредственно главным моментом в проектировании сети является выбор способов передачи данных. Способ передачи определяется сетевой технологией, на основе которой построена ЛВС. Оптимальным решением для данной сети являются технологии Fast Ethernet\footnote{Fast Ethernet (FE) — общее название для набора стандартов передачи данных в компьютерных сетях по технологии Ethernet со скоростью до 100 Мбит/с.} и Wi-Fi. Они полностью подходят для данной ЛВС по скорости передачи данных, безопасности передачи информации, обратной совместимости с предшествующим им технологиям. Для построения сети достаточно иметь по одному сетевому адаптеру поддерживающие стандарты Wi-Fi для каждого компьютера и один маршрутизатор. Данные технологии позволяют иметь ЛВС хорошую расширяемость, низкую стоимость, простота настройки и эксплуатации. Работа стандартов Wi-Fi основана на передаче идентификатора сети SSID с помощью пакетов на скорости 0,1 Мбит/с каждые 100 мс. Потому 0,1 Мбит/с --- наименьшая скорость передачи данных Wi-Fi. С помощью SSID клиент может выяснить о возможности подключения к точке доступа.
Защита канала передачи данных с помощью Wi-Fi основана на методах шифрования:
\begin{itemize}
\item WEP\footnote{Wired Equaivalent Privacy} --- это первый стандарт защиты Wi-Fi, на самом деле не дает защиты по сравнению с проводными сетями, так как имеет множество уязвимостей и взламывается множеством разных способов, что из-за расстояния покрываемого передатчиком, делает данные более уязвимыми. Данный протокол обеспечивает защиту канала передачи данных только на короткое время, спустя которое любую передачу данных можно взломать вне зависимости от сложности пароля --- пароли в WEP либо 40 либо 104 бита, что есть крайне короткая комбинация. Все эти недостатки имеются из-за времени создания WEP --- конец 90-х годов и потому IEEE\footnote{Институт инженеров электротехники и электроники — IEEE (англ. Institute of Electrical and Electronics Engineers) (I triple E — «Ай трипл и») — международная некоммерческая ассоциация специалистов в области техники, мировой лидер в области разработки стандартов по радиоэлектронике, электротехнике и аппаратному обеспечению вычислительных систем и сетей.} в 2004 году обьявили WEP устаревшим.
\item WPA\footnote{Wi-Fi Protected Access}  --- второе поколение защищенных протоколов Wi-Fi. Совершенно иной уровень защиты каналов данных в сравнении с WEP. Длина пароля произвольная, от 8 до 63 байт. Поддерживает различные алгоритмы шифрования данных после авторизации в сети. WPA в главной степени отличается от WEP тем, что шифрует данные каждого клиента по отдельности --- после авторизации генерируется временный ключ PTK который используется для кодирования передачи данных конкретного клиента.
\item WPS\footnote{Wi-Fi Protected Setup} --- протокол, позволяющий для авторизации вместо пароля использовать кнопку на роутере для подключения. При выпуске этого протокола была фундаментальная уязвимость --- WPS позволяет подключиться к точке доступа по 8-ми символьному коду, состоящему из цифр. Но из-за ошибки нужно подобрать всего лишь 4 из них.
\end{itemize}
\end{document}